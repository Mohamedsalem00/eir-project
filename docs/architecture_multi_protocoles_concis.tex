\documentclass[11pt]{article}
\usepackage[utf8]{inputenc}
\usepackage[french]{babel}
\usepackage{geometry}
\usepackage{listings}
\usepackage{xcolor}
\usepackage{graphicx}
\usepackage{float}
\usepackage{hyperref}
\geometry{margin=2cm}

% Configuration des liens hypertexte
\hypersetup{
    colorlinks=true,
    linkcolor=blue,
    filecolor=magenta,
    urlcolor=cyan,
    pdftitle={EIR Multi-Protocoles : Architecture Technique},
    pdfauthor={Mohamed Salem Khyarhoum},
    pdfsubject={Documentation technique EIR},
    pdfkeywords={EIR, Multi-protocoles, SS7, Diameter, REST, DMS},
    bookmarks=true,
    bookmarksopen=true,
    bookmarksnumbered=true
}

\lstset{
    basicstyle=\ttfamily\small,
    backgroundcolor=\color{gray!10},
    frame=single,
    breaklines=true
}

\title{EIR Multi-Protocoles : Architecture Technique}
\author{Mohamed Salem Khyarhoum}
\date{Août 2025}

\begin{document}

\maketitle

\tableofcontents
\newpage

\section{Vue d'ensemble}

Application EIR (Equipment Identity Register) avec support multi-protocoles pour la vérification IMEI :
\begin{itemize}
    \item \textbf{REST/HTTP} : Applications web/mobile
    \item \textbf{SS7/MAP} : Réseaux 2G/3G (MSC/VLR)
    \item \textbf{Diameter} : Réseaux 4G/LTE (MME/SGSN)
    \item \textbf{DMS Sync} : Synchronisation systèmes externes
\end{itemize}

\section{Architecture}

\textbf{Composants principaux :}
\begin{itemize}
    \item \textbf{Protocol Dispatcher} : Routage des requêtes
    \item \textbf{Handlers spécialisés} : REST, SS7, Diameter, DMS
    \item \textbf{Configuration dynamique} : YAML rechargeable
    \item \textbf{Base PostgreSQL} : Stockage IMEI
    \item \textbf{Service d'audit} : Journalisation complète
\end{itemize}

\subsection{Diagramme de Séquence}

\begin{figure}[H]
\centering
\includegraphics[width=0.9\textwidth]{MSC_Extended.png}
\caption{Diagramme de séquence des interactions multi-protocoles}
\label{fig:msc_sequence}
\end{figure}

\subsection{Architecture Système}

\begin{figure}[H]
\centering
\includegraphics[width=0.75\textwidth]{architecture_simple.png}
\caption{Architecture générale du système EIR multi-protocoles}
\label{fig:architecture_system}
\end{figure}

\textbf{Flux de traitement :}
\begin{enumerate}
    \item Réception de la requête (protocole spécifique)
    \item Routage par le dispatcher central
    \item Traitement par le handler approprié
    \item Validation IMEI en base de données
    \item Retour de la réponse (format protocole)
\end{enumerate}

\section{Configuration}

\begin{lstlisting}[language=yaml,caption=config/protocols.yml]
enabled_protocols:
  rest: true
  ss7: true  
  diameter: true

timeouts:
  rest: 30
  ss7: 10
  diameter: 60

endpoints:
  ss7:
    sccp_address: "1234"    # Point Code SCCP
    gt: "33123456789"       # Global Title
  diameter:
    host: "0.0.0.0"
    port: 3868              # Port IANA standard
    realm: "eir.domain.com"
\end{lstlisting}

\section{Utilisation}

\subsection{Vérification IMEI}
\begin{lstlisting}[caption=Requêtes par protocole]
# REST
curl -X POST "http://localhost:8000/verify_imei?protocol=rest" \
     -H "Content-Type: application/json" \
     -d '{"imei": "123456789012345"}'

# SS7 (simulation)
curl -X POST "http://localhost:8000/verify_imei?protocol=ss7" \
     -H "Content-Type: application/json" \
     -d '{"imei": "123456789012345", "msisdn": "33123456789"}'

# Diameter (simulation)
curl -X POST "http://localhost:8000/verify_imei?protocol=diameter" \
     -H "Content-Type: application/json" \
     -d '{"imei": "123456789012345", "session_id": "abc123"}'
\end{lstlisting}

\subsection{Synchronisation DMS}
\begin{lstlisting}[caption=Synchronisation d'appareils]
curl -X POST "http://localhost:8000/sync-device" \
     -H "Content-Type: application/json" \
     -d '{
       "appareils": [
         {
           "imei": "123456789012345",
           "marque": "Samsung",
           "modele": "Galaxy S21",
           "statut": "actif"
         }
       ],
       "sync_mode": "upsert",
       "source_system": "DMS_External"
     }'

# Réponse
{
  "statut": "succès",
  "appareils_synchronisés": 1,
  "créés": 1,
  "mis_à_jour": 0,
  "erreurs": 0,
  "id_sync": "uuid-12345",
  "durée_ms": 62.56
}
\end{lstlisting}

\section{Déploiement}

\begin{lstlisting}[caption=Docker Compose]
# Démarrage complet
docker-compose up -d

# Vérification des protocoles
curl http://localhost:8000/protocols/status

# Logs en temps réel
docker logs eir_web -f
\end{lstlisting}

\section{Avantages}

\begin{itemize}
    \item \textbf{Flexibilité} : Support de tous types d'équipements réseau
    \item \textbf{Configuration dynamique} : Pas de redémarrage nécessaire
    \item \textbf{Synchronisation DMS} : Intégration systèmes externes
    \item \textbf{Audit complet} : Traçabilité de toutes les opérations
    \item \textbf{Performance} : Architecture optimisée
\end{itemize}

\section{Tests}

\begin{lstlisting}[caption=Script de test complet]
#!/bin/bash
# Statut global
curl http://localhost:8000/protocols/status

# Test vérification IMEI
curl -X POST "http://localhost:8000/verify_imei?protocol=rest" \
     -H "Content-Type: application/json" \
     -d '{"imei": "123456789012345"}'

# Test synchronisation
curl -X POST "http://localhost:8000/sync-device" \
     -H "Content-Type: application/json" \
     -d '{
       "appareils": [{"imei": "987654321098765", "marque": "Apple", "modele": "iPhone 14"}],
       "sync_mode": "upsert",
       "source_system": "DMS_Test"
     }'
\end{lstlisting}

\end{document}
