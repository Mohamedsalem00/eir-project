\documentclass[a4paper,12pt]{article}
\usepackage[utf8]{inputenc}
\usepackage[french]{babel}
\usepackage{geometry}
\usepackage{hyperref}
\usepackage{graphicx}
\geometry{margin=2.5cm}
\title{Cahier des Charges}
\author{Projet de vérification IMEI}
\date{}

\begin{document}

\maketitle

\tableofcontents
\newpage

\section{Introduction}
Ce projet vise à développer une API et une interface web permettant la vérification de l'authenticité des numéros IMEI des téléphones mobiles. Il a pour objectif principal de lutter contre l'importation et l'utilisation de téléphones non conformes ou volés.

\section{Objectifs du Projet}
\begin{itemize}
    \item Vérifier si un IMEI est valide et connu.
    \item Identifier la provenance d'un appareil.
    \item Associer plusieurs téléphones et cartes SIM à un utilisateur.
    \item Tracer les recherches effectuées.
\end{itemize}

\section{Architecture Générale}
L'architecture suit le modèle client-serveur :
\begin{itemize}
    \item Le client interroge l’API en fournissant un numéro IMEI (et éventuellement un ICCID).
    \item L’API traite la requête, interroge la base de données locale, puis retourne le statut de l'appareil.
    \item Si l’IMEI est inconnu, une requête est faite à une API externe (ex : GSMA), et les données sont mises en cache localement.
\end{itemize}

\section{Fonctionnalités Attendues}
\begin{itemize}
    \item Recherche d’un numéro IMEI pour vérifier sa validité et son statut.
    \item Affichage des détails d’un appareil associé à un IMEI.
    \item Recherche secondaire vers une API externe si IMEI non trouvé localement.
    \item Mise en cache locale des résultats externes.
    \item Enregistrement d’un nouvel appareil.
    \item Liaison d’un appareil (IMEI) ou d’une carte SIM (ICCID) à un utilisateur.
    \item Authentification sécurisée via JWT.
    \item Historique des recherches et actions utilisateurs.
    \item Notifications par email ou SMS lors d’actions critiques.
    \item Journalisation pour audit.
\end{itemize}

\section{Architecture de la Base de Données}

\subsection{Choix de la structure}
Une base relationnelle PostgreSQL a été choisie pour sa robustesse, sa normalisation, et sa gestion des relations.

\subsection{Schéma simplifié}

\begin{itemize}
    \item \textbf{utilisateurs} : (id, nom, email, mot de passe, numéro téléphone, rôle)
    \item \textbf{telephones} : (id, imei, marque, modèle, statut, date ajout, pays origine)
    \item \textbf{sims} : (id, iccid, opérateur, date activation)
    \item \textbf{utilisateur\_telephone} : (id, utilisateur\_id, telephone\_id, date liaison)
    \item \textbf{utilisateur\_sim} : (id, utilisateur\_id, sim\_id, date liaison)
    \item \textbf{historique\_recherches} : (id, utilisateur\_id, imei recherché, date recherche)
    \item \textbf{audit\_logs} : (id, user\_id, action, table\_name, record\_id, timestamp, details)
    \item \textbf{notifications} : (id, utilisateur\_id, type, message, statut, date\_envoi)
\end{itemize}


\subsection{Champ important : ICCID}
Chaque carte SIM est identifiée de façon unique par un ICCID (Integrated Circuit Card Identifier). Il permet de suivre l’usage et les changements de téléphone d’un utilisateur.

\subsection{Évolution}
\begin{itemize}
    \item Intégration d’une API externe (ex : GSMA) pour recherche étendue.
    \item Réattribution dynamique des appareils/SIM.
    \item Alertes lors de changement d’équipement.
\end{itemize}

\subsection{Mise en cache des données externes}
Pour réduire la charge et la latence, tout IMEI inconnu trouvé via API externe est enregistré localement pour les futures recherches.

\subsection{Système de Notification}

Le système de notification permet d’informer les utilisateurs par différents canaux selon le contexte. Chaque notification contient un type, un message, un statut, et une date d’envoi.

\begin{itemize}
    \item \textbf{Type} : 
    \begin{itemize}
        \item \textbf{email} : confirmations, alertes de sécurité.
        \item \textbf{sms} : alertes urgentes, authentification à deux facteurs (2FA).
    \end{itemize}

    \item \textbf{Statut} :
    \begin{itemize}
        \item \textbf{envoyé} : notification transmise avec succès.
        \item \textbf{échoué} : erreur lors de l’envoi.
        \item \textbf{en attente} : notification en file d’attente.
    \end{itemize}
    
    \item \textbf{Intégration possible} : SendGrid, Twilio, Firebase, etc.
\end{itemize}

\subsection{Alimentation de la Base de Données (Import / Export)}
\begin{itemize}
    \item \textbf{Importation} :
    \begin{itemize}
        \item Fichiers CSV ou JSON contenant des IMEI.
        \item Validation des doublons.
        \item Mise à jour automatique.
    \end{itemize}
    \item \textbf{Exportation} :
    \begin{itemize}
        \item Exports filtrables (statut, utilisateur, date...).
        \item Formats : CSV / JSON.
    \end{itemize}
\end{itemize}

\section{Utilisateurs}

\subsection{Types d'utilisateurs}
\begin{itemize}
    \item \textbf{Administrateur} : contrôle complet du système.
    \item \textbf{Utilisateur Authentifié} : accès complet à ses propres données.
    \item \textbf{Visiteur Anonyme} : 
    \begin{itemize}
        \item Aucun compte requis - pas stocké en base de données.
        \item Recherche IMEI limitée (rate limiting).
        \item Informations limitées dans les résultats de recherche.
        \item Pas d'historique ni de fonctionnalités avancées.
    \end{itemize}
\end{itemize}

\section{Sécurité}
\begin{itemize}
    \item Authentification JWT.
    \item Hachage des mots de passe (ex : bcrypt).
    \item Limitation des requêtes (rate limiting).
    \item Journalisation des accès.
\end{itemize}

\section{Technologies Utilisées}
\begin{itemize}
    \item \textbf{Backend} : FastAPI (Python)
    \item \textbf{Base de données} : PostgreSQL
    \item \textbf{Conteneurisation} : Docker / Docker Compose
    \item \textbf{Authentification} : JWT
    \item \textbf{Frontend} : HTML/CSS/JS ou équivalent
\end{itemize}

\section{Livrables}
\begin{itemize}
    \item Code source (API + interface)
    \item Fichier \texttt{docker-compose.yml}
    \item Script SQL de création de la base
    \item Documentation technique et utilisateur
    \item Présentation finale
\end{itemize}

\end{document}
