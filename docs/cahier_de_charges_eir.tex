\documentclass[12pt,a4paper]{article}
\usepackage[utf8]{inputenc}
\usepackage[T1]{fontenc}
\usepackage[french]{babel}
\usepackage{geometry}
\usepackage{graphicx}
\usepackage{hyperref}
\usepackage{enumitem}
\geometry{margin=2.5cm}

\title{Cahier des Charges du Projet EIR (Enregistrement des IMEI et RIB)}
\author{Mohamed Salem Khyarhoum}
\date{\today}

\begin{document}

\maketitle
\tableofcontents
\newpage

\section{Introduction}
Ce projet consiste à développer une plateforme web sécurisée permettant de vérifier l’état d’un appareil mobile (volé ou non) à partir de son numéro IMEI. La plateforme vise à assister les utilisateurs, opérateurs et autorités dans la lutte contre la contrefaçon et le trafic des téléphones portables.

\section{Objectifs du Projet}
\begin{itemize}
    \item Développement d’un site web responsive.
    \item Vérification de l’IMEI via une base de données centralisée.
    \item Authentification des utilisateurs avec JWT.
    \item Implémentation d’une API REST sécurisée.
    \item Gestion de l’historique de requêtes.
    \item Infrastructure conteneurisée (Docker).
\end{itemize}

\section{Technologies Utilisées}
\begin{itemize}
    \item \textbf{Backend} : FastAPI (Python)
    \item \textbf{Base de Données} : PostgreSQL
    \item \textbf{Frontend} : HTML, CSS (ou simple template Bootstrap si nécessaire)
    \item \textbf{Sécurité} : JWT, HTTPS, validation des entrées
    \item \textbf{Test} : Postman pour tester l’API
    \item \textbf{Déploiement} : Docker
\end{itemize}

\section{Architecture Générale}
\begin{itemize}
    \item Un serveur Linux (hébergé ou local via Docker)
    \item API REST exposée (FastAPI)
    \item Base de données PostgreSQL sécurisée
    \item Interface utilisateur simple pour soumettre un IMEI
    \item Logs d’accès pour audit
\end{itemize}

\section{Fonctionnalités Attendues}
\begin{itemize}
    \item Soumission d’un IMEI et vérification dans la base
    \item Authentification par jeton (JWT)
    \item Gestion des rôles (admin, utilisateur)
    \item Historique des consultations
    \item API REST pour intégration éventuelle avec d’autres systèmes
\end{itemize}

\section{Utilisateurs}
\subsection{Types d’utilisateurs}
\begin{itemize}
    \item \textbf{Administrateur} :
        \begin{itemize}
            \item Accès complet à la base IMEI
            \item Possibilité d’ajouter/supprimer/modifier des entrées
            \item Gestion des comptes utilisateurs
        \end{itemize}
    \item \textbf{Utilisateur Authentifié} :
        \begin{itemize}
            \item Peut vérifier un IMEI
            \item Accès illimité (selon quota défini) à la recherche IMEI.
            \item Accède à son historique de recherche
            \item Les actions sont journalisées à des fins d’audit.
        \end{itemize}
    \item \textbf{Visiteur Anonyme} :
        \begin{itemize}
            \item Accès limité à un seul formulaire de recherche IMEI par jour.
            \item N’a pas accès à l’historique ni aux détails étendus.
        \end{itemize}
\end{itemize}

\section{Sécurité}
\begin{itemize}
    \item Authentification par JWT
    \item Sécurisation des communications via HTTPS
    \item Protection contre les injections SQL et autres attaques courantes
\end{itemize}

\section{Livrables}
\begin{itemize}
    \item Code source (API + interface)
    \item Fichier Dockerfile + docker-compose
    \item Documentation d’utilisation
    \item Script SQL de création de la base
\end{itemize}


\end{document}
