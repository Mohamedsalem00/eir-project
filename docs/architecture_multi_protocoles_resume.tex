\documentclass[11pt]{article}
\usepackage[utf8]{inputenc}
\usepackage[french]{babel}
\usepackage{geometry}
\usepackage{listings}
\usepackage{xcolor}
\geometry{margin=2cm}

\lstset{
    basicstyle=\ttfamily\small,
    backgroundcolor=\color{gray!10},
    frame=single,
    breaklines=true
}

\title{EIR Multi-Protocoles : Guide Technique}
\author{Mohamed Salem Khyarhoum}
\date{Août 2025}

\begin{document}

\maketitle

\section{Vue d'ensemble}

Application EIR (Equipment Identity Register) avec support multi-protocoles pour la vérification IMEI :
\begin{itemize}
    \item \textbf{REST/HTTP} : Applications web/mobile
    \item \textbf{SS7/MAP} : Réseaux 2G/3G (MSC/VLR)
    \item \textbf{Diameter} : Réseaux 4G/LTE (MME/SGSN)
\end{itemize}

\section{Configuration}

\begin{lstlisting}[language=yaml,caption=config/protocols.yml]
enabled_protocols:
  rest: true
  ss7: true  
  diameter: true

timeouts:
  rest: 30
  ss7: 10
  diameter: 60

endpoints:
  ss7:
    sccp_address: "1234"    # Point Code SCCP
    gt: "33123456789"       # Global Title
  diameter:
    host: "0.0.0.0"
    port: 3868              # Port IANA standard
    realm: "eir.domain.com"
\end{lstlisting}

\section{Utilisation}

\subsection{Test REST}
\begin{lstlisting}[caption=Requête REST]
curl -X POST "http://localhost:8000/verify_imei?protocol=rest" \
     -H "Content-Type: application/json" \
     -d '{"imei": "123456789012345"}'
\end{lstlisting}

\subsection{Test SS7 (simulation)}
\begin{lstlisting}[caption=Requête SS7]
curl -X POST "http://localhost:8000/verify_imei?protocol=ss7" \
     -H "Content-Type: application/json" \
     -d '{"imei": "123456789012345", "msisdn": "33123456789"}'
\end{lstlisting}

\subsection{Test Diameter (simulation)}
\begin{lstlisting}[caption=Requête Diameter]
curl -X POST "http://localhost:8000/verify_imei?protocol=diameter" \
     -H "Content-Type: application/json" \
     -d '{"imei": "123456789012345", "session_id": "abc123"}'
\end{lstlisting}

\section{Architecture}

\textbf{Composants principaux :}
\begin{itemize}
    \item \textbf{Protocol Dispatcher} : Routage des requêtes
    \item \textbf{Handlers spécialisés} : REST, SS7, Diameter
    \item \textbf{Configuration dynamique} : YAML rechargeable
    \item \textbf{Base PostgreSQL} : Stockage IMEI
    \item \textbf{Cache Redis} : Performance
\end{itemize}

\textbf{Flux de traitement :}
\begin{enumerate}
    \item Réception de la requête (protocole spécifique)
    \item Routage par le dispatcher central
    \item Traitement par le handler approprié
    \item Validation IMEI en base de données
    \item Retour de la réponse (format protocole)
\end{enumerate}

\section{Déploiement}

\begin{lstlisting}[caption=Docker Compose]
# Démarrage complet
docker-compose up -d

# Vérification des protocoles
curl http://localhost:8000/protocols/status

# Logs en temps réel
docker-compose logs -f eir_web
\end{lstlisting}

\section{Avantages}

\begin{itemize}
    \item \textbf{Flexibilité} : Support de tous types d'équipements réseau
    \item \textbf{Évolutivité} : Ajout facile de nouveaux protocoles
    \item \textbf{Performance} : Cache Redis et optimisations
    \item \textbf{Monitoring} : Journalisation complète et métriques
    \item \textbf{Sécurité} : Authentification par protocole
\end{itemize}

\section{Configuration Avancée}

\subsection{Activation/Désactivation}
Modifier \texttt{config/protocols.yml} et recharger :
\begin{lstlisting}
# Les changements sont automatiques !
# Aucun redémarrage nécessaire
\end{lstlisting}

\subsection{Monitoring}
\begin{lstlisting}
# Statut des protocoles
GET /protocols/status

# Métriques de performance  
GET /metrics

# Logs d'audit
docker-compose logs eir_web | grep "AUDIT"
\end{lstlisting}

\section{Références}

\begin{itemize}
    \item Documentation complète : \texttt{docs/architecture\_multi\_protocoles.tex}
    \item Diagrammes UML : \texttt{docs/uml/}
    \item Tests API : \texttt{test/}
    \item Scripts de gestion : \texttt{scripts/}
\end{itemize}

\end{document}
